%%%%%%%%%%%%%%%%%%%%%%%%%%%%%%%%%%%%%%%%%
% Masters/Doctoral Thesis 
% LaTeX Template
% Version 1.42 (19/1/14)
%
% This template has been downloaded from:
% http://www.latextemplates.com
%
% Original authors:
% Steven Gunn 
% http://users.ecs.soton.ac.uk/srg/softwaretools/document/templates/
% and
% Sunil Patel
% http://www.sunilpatel.co.uk/thesis-template/
%
% License:
% CC BY-NC-SA 3.0 (http://creativecommo\textit{•}ns.org/licenses/by-nc-sa/3.0/)
%
% Note:
% Make sure to edit document variables in the Thesis.cls file
%
%%%%%%%%%%%%%%%%%%%%%%%%%%%%%%%%%%%%%%%%%

%----------------------------------------------------------------------------------------
%	PACKAGES AND OTHER DOCUMENT CONFIGURATIONS
%----------------------------------------------------------------------------------------

\documentclass[11pt, a4paper]{Thesis} % Paper size, default font size and one-sided paper

\graphicspath{{Pictures/}} % Specifies the directory where pictures are stored

\usepackage{comment}
\usepackage{pdfpages}
\usepackage{hyperref} 
\usepackage{array}
\usepackage{graphicx}

\usepackage[square, numbers, comma, sort&compress]{natbib} % Use the natbib reference package - read up on this to edit the reference style; if you want text (e.g. Smith et al., 2012) for the in-text references (instead of numbers), remove 'numbers' 
\hypersetup{urlcolor=blue, colorlinks=true} % Colors hyperlinks in blue - change to black if annoying
\title{\ttitle} % Defines the thesis title - don't touch this

\let\cleardoublepage\clearpage

\usepackage{titlesec}
\titleformat{\chapter}[hang]
{\normalfont\huge\bfseries}{\chaptertitlename\ \thechapter:}{1em}{}
\titlespacing*{\chapter}{0cm}{-\topskip}{0pt}[0pt]
\titlespacing{\section}{0pt}{*0}{*0}
\titlespacing{\subsection}{0pt}{*0}{*0}
\titlespacing{\subsubsection}{0pt}{*0}{*0}

\begin{document}

\frontmatter % Use roman page numbering style (i, ii, iii, iv...) for the pre-content pages

\setstretch{1.5} % Line spacing of 1.5

% Define the page headers using the FancyHdr package and set up for one-sided printing
\fancyhead{} % Clears all page headers and footers
\rhead{\thepage} % Sets the right side header to show the page number
\lhead{} % Clears the left side page header

\pagestyle{fancy} % Finally, use the "fancy" page style to implement the FancyHdr headers

\newcommand{\HRule}{\rule{\linewidth}{0.5mm}} % New command to make the lines in the title page

% PDF meta-data
\hypersetup{pdftitle={\ttitle}}
\hypersetup{pdfsubject=\subjectname}
\hypersetup{pdfauthor=\authornames}
\hypersetup{pdfkeywords=\keywordnames}

%----------------------------------------------------------------------------------------
%	TITLE PAGE
%----------------------------------------------------------------------------------------

\begin{titlepage}
\begin{center}

\textsc{\LARGE \univname}\\[1.5cm] % University name
\textsc{\Large Final Year Project}\\[0.5cm] % Thesis 
\Large \textit{Mid Project Report}\\[0.5cm]

\HRule \\[0.4cm] % Horizontal line
{\huge \bfseries \ttitle}\\[0.4cm] % Thesis title
\HRule \\[1.5cm] % Horizontal line
 
\begin{minipage}{0.4\textwidth}
\begin{flushleft} \large
\emph{Author:}\\
\href{https://engineering.leeds.ac.uk/cgi-bin/sis/comp/ts/student.cgi?student=200542912}{\authornames} % Author name - remove the \href bracket to remove the link
\end{flushleft}
\end{minipage}
\begin{minipage}{0.4\textwidth}
\begin{flushright} \large
\emph{Supervisor:} \\
\href{https://www.engineering.leeds.ac.uk/people/faculty/staff/r.s.kwan}{\supname} % Supervisor name - remove the \href bracket to remove the link  
\end{flushright}
\end{minipage}\\[3cm]
 
\large \textit{A final year project report submitted in fulfilment of the requirements\\ for the degree of \degreename}\\[0.3cm] % University requirement text
\textit{in the}\\[0.4cm]
\deptname\\[2cm] % Research group name and department name
 
{\large \today}\\[4cm] % Date
%\includegraphics{Logo} % University/department logo - uncomment to place it
 
\vfill
\end{center}

\end{titlepage}

\begin{comment}
%----------------------------------------------------------------------------------------
%	QUOTATION PAGE\usepackage{comment}
%----------------------------------------------------------------------------------------

\pagestyle{empty} % No headers or footers for the following pages

\null\vfill % Add some space to move the quote down the page a bit

\textit{``Thanks to my solid academic training, today I can write hundreds of words on virtually any topic without possessing a shred of information, which is how I got a good job in journalism."}

\begin{flushright}
Dave Barry
\end{flushright}

\vfill\vfill\vfill\vfill\vfill\vfill\null % Add some space at the bottom to position the quote just right

\clearpage  Start a new page

\end{comment}
%----------------------------------------------------------------------------------------
%	ABSTRACT PAGE
%----------------------------------------------------------------------------------------

\addtotoc{Abstract} % Add the "Abstract" page entry to the Contents

\abstract{\addtocontents{toc}{\vspace{1em}} % Add a gap in the Contents, for aesthetics

This project aims to allow more communication with outpatients in regards to their appointments. It also aims to tackle various problems such as attendance, human opportunity costs and patient satisfaction. This will be achieved using emerging technology to bridge the current communication gap between patients and the scheduling systems used.

\clearpage % Start a new page

\begin{comment}
%----------------------------------------------------------------------------------------
%	ACKNOWLEDGEMENTS
%----------------------------------------------------------------------------------------

\setstretch{1.3} % Reset the line-spacing to 1.3 for body text (if it has changed)

\acknowledgements{\addtocontents{toc}{\vspace{1em}} % Add a gap in the Contents, for aesthetics

The acknowledgements and the people to thank go here, don't forget to include your project advisor\ldots
}
\clearpage % Start a new page

\end{comment}

%----------------------------------------------------------------------------------------
%	LIST OF CONTENTS/FIGURES/TABLES PAGES
%----------------------------------------------------------------------------------------

\pagestyle{fancy} % The page style headers have been "empty" all this time, now use the "fancy" headers as defined before to bring them back

\lhead{\emph{Contents}} % Set the left side page header to "Contents"
\tableofcontents % Write out the Table of Contents

\lhead{\emph{List of Figures}} % Set the left side page header to "List of Figures"
\listoffigures % Write out the List of Figures

\begin{comment}
\lhead{\emph{List of Tables}} % Set the left side page header to "List of Tables"
\listoftables % Write out the List of Tables
\end{comment}

%----------------------------------------------------------------------------------------
%	ABBREVIATIONS
%----------------------------------------------------------------------------------------

\clearpage % Start a new page

\setstretch{1.5} % Set the line spacing to 1.5, this makes the following tables easier to read

\lhead{\emph{Abbreviations}} % Set the left side page header to "Abbreviations"
\listofsymbols{ll} % Include a list of Abbreviations (a table of two columns)
{
\textbf{NHS} & \textbf{N}ational \textbf{H}ealth \textbf{S}ervice \\
\textbf{APP} & Software \textbf{APP}lication \\
\textbf{API} & \textbf{A}lication \textbf{P}rogramming \textbf{I}nterface \\
\textbf{MVC} & \textbf{M}odel \textbf{V}iew \textbf{C}ontroller \\
\textbf{XML} & \textbf{E}xtensible \textbf{M}arkup \textbf{L}anguage \\
\textbf{JSON} & \textbf{J}ava\textbf{S}cript \textbf{O}bject \textbf{N}otation language \\
\textbf{SQL} & \textbf{S}tructured \textbf{Q}uery \textbf{L}anguage \\
\textbf{LINQ} & \textbf{L}anguage \textbf{IN}tegrated \textbf{Q}uery\\
\textbf{ER} & \textbf{E}ntity \textbf{R}elationship\\
\textbf{GCM} & \textbf{G}oogle \textbf{C}loud \textbf{M}essaging\\
}

%----------------------------------------------------------------------------------------
%	THESIS CONTENT - CHAPTERS
%----------------------------------------------------------------------------------------

\mainmatter % Begin numeric (1,2,3...) page numbering

\pagestyle{fancy} % Return the page headers back to the "fancy" style

% Include the chapters of the thesis as separate files from the Chapters folder
% Uncomment the lines as you write the chapters

\setstretch{1.5}

% Chapter 1

\chapter{Project Overview} % Main chapter title

\label{Chapter1} % For referencing the chapter elsewhere, use \ref{Chapter1} 

\lhead{Chapter 1. \emph{Project Overview}} % This is for the header on each page - perhaps a shortened title

%----------------------------------------------------------------------------------------

\section{Project Aim}

The aim of this project is to develop a system that allows more communication with outpatients in regards to their appointments, in an attempt to: 

\begin{itemize}
  \item reduce the amount of resources wasted in the event of no-shows and cancellations
  \item reducing the human cost that is required to manage appointments
  \item increase the user experience when making and managing an appointment
\end{itemize}

This project will not aim to implement any scheduling algorithms, but rather create a system that will support current algorithms and possibly shape future algorithms. For this project, I will use a preordained scenario to show off the different features of the system.

It will also not aim to replace the current scheduling system entirely, but act as an alternative to allow willing patients more direct control and easier access to information regarding their appointments. 

%----------------------------------------------------------------------------------------

\section{Objectives}

The objectives of this project is to:

\begin{itemize}
  \item Collect relevant background data about the problem domain
  \item Identify requirements necessary to address the problem
  \item Design a server-client system that implements these requirements
  \item Prototype a Server that communicates directly with multiple patients and a predetermined scheduling algorithm
  \item Prototype a Client Application (Smart-phone Application) that allows patients to interface with the server
  \item Test the systems functionality based on usability and performance
  \item Evaluate the success of the system in regards to improving user satisfaction, reducing human management resources and reducing appointment wastage
\end{itemize}

%----------------------------------------------------------------------------------------

\section{Minimum Requirements}

The minimum requirements of this project is to:

\begin{itemize}
	\item A working prototype smart-phone application that:
	\begin{itemize}
		\item Connects to a prototype server
		\item Allows the user to view information on their scheduled medical appointments
		\item Gives the user information such as location and map instructions, doctor's name, any perquisite tasks the user must undertake prior to the appointment
		\item Reminds the user about the appointment
		\item Receives appointment updates from the server
	\end{itemize}
	\item A working prototype server application that:
		\begin{itemize}
			\item Connects to multiple clients
			\item Interfaces with a dumb appointment scheduler
			\item Notifies clients of changes to the schedule
			\item Uses client information to optimise the scheduling process
			\item Offers cancelled appointments to other clients
		\end{itemize}
\end{itemize}

%----------------------------------------------------------------------------------------

\section{Extensions}

The possible extensions are:

\begin{itemize}
	\item Design a website interface for the system
	\item Investigate and test security issues
	\item App gives location information and integrates with google maps
	\item Additional App or website to be used by doctors
\end{itemize}

%----------------------------------------------------------------------------------------

\section{Deliverables}

The deliverables are:

\begin{itemize}
	\item Server application
	\item Client application
	\item Report and evaluation results
\end{itemize}

%----------------------------------------------------------------------------------------

\section{Relevance to Degree Modules}

This project uses knowledge and techniques gained in modules studied as part of my Computer Science degree. The most relevant modules are typically 'Software Systems Engineering', 'Distributed Systems', 'User Adaptive Systems' and 'Human Computer Interaction'.

%----------------------------------------------------------------------------------------
% Chapter Template

\chapter{Project Plan} % Main chapter title

\label{Chapter2} % Change X to a consecutive number; for referencing this chapter elsewhere, use \ref{ChapterX}

\lhead{Chapter 2. \emph{Project Plan}} % Change X to a consecutive number; this is for the header on each page - perhaps a shortened title

%----------------------------------------------------------------------------------------

\section{Schedule}

The schedule was designed as a Gantt chart, in order to highlight key points in the project and ensure that they were completed on time and according to the set deadlines. This schedule is subject to change as the project progresses, as many unforeseen issues may arise. I have tried to estimate tasks based on difficulty, assigning more time to key points where it is likely that problems will occur (such as the implementation phase). Any changes will be shown in an adjusted project schedule.

\begin{figure}[htbp]
	\centering
\includegraphics[width=10cm,height=10cm,keepaspectratio]{Figures/projectgant.png}
		\rule{35em}{0.5pt}
	\caption[Gantt Chart showing the schedule of the project]{Gant Chart showing the schedule of the project}
	\label{fig:projectgant}
\end{figure}

A table was also created to compare objectives with deliverable deadlines.

\begin{figure}[htbp]
	\centering
\includegraphics[width=10cm,height=10cm,keepaspectratio]{Figures/projectplan.png}
		\rule{35em}{0.5pt}
	\caption[Table showing the schedule of the project]{Table showing the schedule of the project}
	\label{fig:projectplan}
\end{figure}

The project has been split four main components, consisting of research, design, implementation and evaluation. These areas are then sub-divided into appropriate sections, targeting individual parts of the system, individual objectives and individual deliverables.

The following sections will explain these components, what parts of the project they will address and how they will be executed successfully.

%----------------------------------------------------------------------------------------

\section{Background Research}

The research component will aim to get a better understanding of the problem, how I can address my aims and objectives and highlighting the current research in the field. This research will aid my own project by highlighting areas that need to be addressed, outlining unforeseen problems and finding technologies that I might be useful to my implementation.

%----------------------------------------------------------------------------------------

\section{Design}

The design component will aim to design the entire system, addressing all objectives whilst making the system both flexible and scalable.

\subsection{Methodologies}

I will choose a software engineering methodology that best fits the development style of the project, explaining how it works and how it is executed successfully. I will also identify the programming languages I will be using, any programming patterns that are useful in the design of the system and Technologies I will use.

\subsection{Flexibility}

Although the system is a prototype, I aim to create it such that it can be adapted in the future. The system will therefore be designed in a way such that additional platforms can be added easily. The system should also be designed in a way that it is feature independent, eliminating dependencies in the code. This will allow more features to be added easily in the future.

\subsection{Scalability}

The system will also be designed to be scalable. It must be able to scale with user demand, be fault tolerant and and appear seamless to the end user, allowing for a smooth service at all times. I will talk about the various problems involved and how I will design the system to overcome them.

\subsection{Problems}

Many problems must be overcome in the implementation phase, such as data security, synchronising data and many more. These problems must be planned in advance, and so I will discuss in detail the problems that the system faces and my proposed solutions to them.

%----------------------------------------------------------------------------------------

\section{Implementation}

The implementation of the system will occur in two separate platforms, the client and the server. These will sometimes overlap as some code will be shared between the client and the server, however for the most part, I will discuss issues unique to each platform separately.

\subsection{Server Platform}

For the server platform, I will discuss how I implemented the communication and data storage features, and how it can interface with scheduling software.

\subsection{Client Platform}

For the client platform, I will discuss how the user interface is created, how I improve performance and how well it interfaces with the mobile device.

%----------------------------------------------------------------------------------------

\section{Evaluation}

The evaluation component will show how well my prototype system does in solving my aims and objectives identified in the project. I will analyse the results of user evaluation methods to evaluate the prototype, discussing the results and identifying key parts of my solution that need to be improved. I plan to evaluate on the following questions:

\begin{itemize}
	\item How good is the User Interface?
	\item Does the application work?
	\item Is the application buggy, does it ever fail?
	\item Is it missing key features?
	\item How can it be improved?
	\item Would the user find the application useful?
\end{itemize}

\section{Conclusion}

From this evaluation, I will formulate a conclusion, reviewing my design choices, how accurate my expectations were and discussing future extensions that this project could inspire.

%----------------------------------------------------------------------------------------
 
% Chapter Template

\chapter{Problem Description and Background Research} % Main chapter title

\label{Chapter3} % Change X to a consecutive number; for referencing this chapter elsewhere, use \ref{ChapterX}

\lhead{Chapter 3. \emph{Problem Description and Background Research}} % Change X to a consecutive number; this is for the header on each page - perhaps a shortened title

%----------------------------------------------------------------------------------------

\section{Problem Description}

Medical appointment scheduling is a complex problem; patients often come with different backgrounds and personal schedules, requiring different treatment and different urgency, some even requiring support in getting to the appointments. Patients sometimes have a need to cancel their appointments or simply do not turn up, which can lead to a waste in resources if the appointment slot is not then assigned to another patient. Often, sessions can overrun, requiring more time per patient than is estimated, and so the following appointments are delayed. Clinics also reschedule appointments regularly, as new patients requiring urgent medical attention become a higher priority. This results in appointments being dynamic, often the time and date of the actual appointment is different from what was originally planned.

Dynamic scheduling leads to many issues. The problem of scheduling appointments becomes far more complex, which in turn requires more staff resources to manage the appointments.

Communication also becomes a problem, as patients need to be informed about all changes to the original schedule. Often, this results in a lower patient satisfaction and a higher chance of appointment cancellations.

These often contribute to longer waiting times and a lack of patient knowledge about their appointments, which can make no-shows and further last-minute cancellations more frequent. Often the clinic will not find a patient to take the free appointment slot, and these resources are wasted.

\subsection{Wasted Resources}

Many resources are wasted through appointment no-shows and cancellations. Research shows that the longer a patient must wait between making the appointment and the actual appointment date, the more likely it is that they will either cancel or not turn up\cite{Gallucci}. Although we can see that there is a relationship between the length of time that a patient must wait for an appointment and the cancellation risk, it is important to understand why.

The most common reasons why patients do not show up was collected through patient questionnaires as can be seen in the figure below.

\begin{figure}[htbp]
	\centering
		\includegraphics[width=10cm,height=10cm,keepaspectratio]{Figures/MissedAppointmentsStoneEtAl.png}
		\rule{35em}{0.5pt}
	\caption[Factors contributing to non-attendance according to a patient questionaire - \cite{Stone}]{Factors contributing to non-attendance according to a patient questionaire - \cite{Stone}}
	\label{fig:NonAttendance}
\end{figure}

This shows that the most common factor is either that the patient forgot or that they were improperly informed by the clinic.

\subsubsection{Patient No-Shows}

Patient no-shows (or non-attendance at outpatient clinics) are a big problem in appointment scheduling and is one of the largest contributors to wasted resources in the NHS. It is estimated  that the financial cost of missed appointments contributes to a loss of £360 million per year\cite{Stone}. Besides the financial costs, it also increases waiting times as that patient must be rescheduled for another appointment, effectively doubling the required resources per patient. 

Because the most common factor contributing to no-shows is the patient forgetting to attend the appointment, it suggests a reminder system could be used effectively to reduce these numbers. Research has shown that telephone and postal reminders can help, but have not proved to be cost effective in the past\cite{Mann}. However, through recently emerging online devices and 'smart' technology, it is possible to provide a low-cost solution to this problem.

Another factor that can encourage non-attendance can be a lack of information known about the appointment. This can introduce a level of uncertainty within the patient, such as knowledge on how to get to their appointment, or  fear about the dangerous/embarrassing factors involved in the appointment\cite{Frankel}.

Research shows that the population that miss appointments are increasingly of a young demographic. Often they fail to understand why the appointments are important, and specifically why it is important to cancel the appointment rather than just not turn up.

Patients have given several reasons for no-shows in studies and questionnaires\cite{Lacy}:

\begin{itemize}
	\item can't get time off work
	\item child-care
	\item lack of transportation or cost
	\item patient felt better or felt too worse to attend the appointment
\end{itemize}

For all these reasons, the no-show is preventable simply by increasing communication with the patient and allowing them either more information about the appointment or making it easier for them to either cancel/reschedule.

\subsubsection{Cancellations}

Often, clinics are required to cancel an outpatients appointment for a variety of reasons, usually due to a lack of resources such as staff or equipment.

Staff time is then wasted getting contact information for the patient and informing them of the cancellation. Time must also be spent corresponding with the patient and agreeing on a suitable replacement appointment.

If cancellations 'occur at the last minute', often the patient is not informed until they reach the hospital. This is often due to the clinic not wanting to bother wasting resources reaching the patient when it is unlikely that they will get a hold of them (the patient may already be in transit or indisposed). This leads to a lower overall user satisfaction as the patient wastes a trip to the hospital, only to find that they no longer have an appointment.

Although cancellations are not ideal, they are better than no-shows in that it is possible for the appointment to be offered to another patient. In some cases however, this is not attempted due to it being too costly and the additional complexity involved in the scheduling process.

By increasing communication with patients through emerging smart technology, it may be possible for these appointments to be quickly rescheduled whilst avoiding the additional costs, even for 'last-minute' cancellations.

\subsubsection{Managing Appointments}

Managing appointments is costly and a large proportion of the work is carried out by individual staff members, rather than an automated system. Several issues arise with this process:

\begin{itemize}
	\item Appointments can only be managed within office hours (typically 9am - 5pm), however this depends on the clinic
	\item Patients can only make appointments over the phone and frequently have to queue to speak to a staff member
	\item Large proportions of staff members must be allocated to the appointments procedure which could be allocated elsewhere
	\item State of the system means that it is hard to analyse and therefore adapt to high demand
	
\end{itemize}

When making appointments, the patient is required to do so either in person or over the phone. This can only occur within office hours, which can conflict with the patients career or personal schedule, leading to a lower user satisfaction. Often patients will have to queue in order to speak to a staff member. When the patient is finally connected, there isn't time for the patient to explain their personal schedule and discuss conflicts, resulting in less user choice and flexibility.

The appointment system in many areas is also very labour-intensive, carried out by individual receptionists using spreadsheets and paper based systems\cite{ApointProcWebsite}. This means that it is often very hard to analyse capacity and demand, identifying bottlenecks or methods to improve them and also makes it very hard to integrate with interactive technology such as smart phone devices and online appointments. Whilst some online systems do exist\cite{C&BWebsite}, they are time consuming, have poor functionality and tend to be only available on few devices\cite{C&BFailure}. 

These problems can be improved by creating an interactive online system that works on many devices. It could not replace the current system entirely, because not all patients will have internet access or smart-devices, but it would provide benefits to patients with internet access such as:

\begin{itemize}
	\item Easier accessibility to making appointments
	\item Possibly offer more appointment flexibility to the patient
	\item More information about the appointment
	\item Relieve demand on the staff that manage the system
	\item More analysis of appointment trends and offer insights into improvement
\end{itemize}

\subsection{Patient Satisfaction and Experience}

Maintaining a high patient satisfaction is the primary goal in appointment scheduling, ultimately because keeping the patients happy leads to less cancellations and less no-shows. This is not an easy task because the demand on the healthcare system is so great, and it is typically faced with many challenges.

\subsubsection{Patient Requirements}

Patients are given a level of responsibility that some may not be used to. For a general outpatient appointment, the NHS requires the patient to do a number of tasks to prepare for the appointment \cite{OutpatientApointWebsite}:

\begin{itemize}
  \item may be required not to eat/drink before the appointment
  \item may be required to bring samples of urine/stool or medicines
  \item may need to bring previous test results
  \item may need to take certain medicines at a certain time period prior to the appointment
  \item should bring maps and other information required for getting to their appointment
\end{itemize}

It has been seen that in previous research conducted on day surgery outpatients, the most likely cause of preventable appointment cancellations (5\% of day surgery appointments) was due to inadequate preparation \cite{Macarthur}. This shows that a large amount of patient cancellations occur simply because patients are expected to find out information about their appointment, transport options and other relevant factors.

\subsubsection{Waiting times}

Another factor that lowers patient satisfaction are waiting times that can occur when the schedule is either running late due to overrunning appointments, or when patients are grouped into time slots.

Patients are frequently grouped together into time slots to simplify the scheduling process (i.e the clinic will expect to have 10 appointments in one hour, so they ask all 10 patients to come at the same time and the appointments occur on a first come first serve basis).

Research suggests that because patients spend increasingly lengthy amounts of time waiting in the clinic for their appointment to start, they feel increasingly amounts of disrespect \cite{Lacy}. This is due to patients being 'left in the dark', with no indication on why their appointment is delayed and why they have to wait.

Through on-line applications and smart devices, we can inform patients about information related to their appointments, any disruptions in the regular service (waiting delays) and a more interactive system that would make the patient feel less disrespect. We will also be able to offer sooner appointments to patients as cancellations occur, which should reduce the waiting times overall.

We can also provide transport information, reminders on when they have to leave and any perquisite requirements that the patient must undertake before leaving for their appointment (such as take medication or bring test results), reducing the likelihood of cancellations and no-shows as the patient is better prepared..

\subsubsection{Patient Participation}

Patient participation is no longer just a goal set by medical commissioners, but a legal obligation. The Health and Social Care Act 2012\cite{HSCA2012} introduced legislation that enables patients(and carers) to participate in the planning, managing and making decisions
about their care and treatment.

The aim of this project targets this participation, engaging patients to have more control and access over their medical care. This system also has the ability to deliver personalised care plans to patients, which will increase the overall patient satisfaction.

Although this system does require patients to have access to the internet and know how to use it, patients without internet in today's world is an increasingly small demographic. The NHS are also launching a program to help disadvantaged people learn how to access the internet and use medical services \cite{timKelsey}, to try and combat these issues. 

%----------------------------------------------------------------------------------------

\section{Choose and Book - An existing online medical appointment service}

An NHS service 'Choose and Book' was launched in 2006, aimed at providing patients with more choice through online appointments\cite{Walford}.

This system is similar to the project area in that it allows patients to create online outpatient appointments, having a choice over which clinic they go to and when they the appointment is booked for.

However, an independent survey of patient's experience using the service in 2008 showed that patients did not receive the degree of choice that the service was designed to deliver\cite{Green}. It has also been widely criticised as being time consuming, over complicated and . An article in 2012\cite{C&BFailure} shows that the system's popularity is diminishing.

Besides the clear flaws in the system such as ease of use and failing to offer more choice, it also fails to target significant areas of the problem description, such as electronic reminders, recycling unused appointments and general appointment information.

%----------------------------------------------------------------------------------------

\section{Conclusion}

Attempts have been made in the past to simplify the appointment management process and take it online, however they fail to hit all of the objectives simply because the platforms were not ready. As smart-devices and their many applications are becoming increasingly popular, it opens up a new gateway to communicate directly with patients and receive quick response times. This makes it much easier to create a dynamic appointment schedule whilst maintaining a high user satisfaction level.

This project will therefore focus on improving the communication and interactivity between patients and the appointment scheduling system; so that more information is available to the patient, there is more chance of reusing free appointment slots, and less staff resources are used in managing appointments so they can be allocated to other areas.

The project will also look at making the appointment creation and rescheduling process easier for both patients and staff, requiring less management resources and offering more platform choices and flexibility.
% Chapter Template

\chapter{System Design and Development} % Main chapter title

\label{Chapter4} % Change X to a consecutive number; for referencing this chapter elsewhere, use \ref{ChapterX}

\lhead{Chapter 4. \emph{System Design and Development}} % Change X to a consecutive number; this is for the header on each page - perhaps a shortened title

%----------------------------------------------------------------------------------------

\section{Introduction}

This chapter will the requirements that the system must fulfil, outline the challenges involved and highlight the key technologies that I will be using.


%----------------------------------------------------------------------------------------

\section{Cross-Platform}

One of the main requirements is running on multiple devices so that the usability of the service is maximised. Although this project will only target the Android platform, it will be designed such that it can be adapted to also run on iOS.

For the sake of simplicity and given the time requirements, this project will also aim to implement the server and client in the same programming language.

\subsection{Mono and Xamarin}

Mono is an open source implementation of C\# that is cross-platform. This allows development of C\# code that can run on Windows, Linux and Mac.

Xamarin is a mobile app development framework built on Mono, targeting the Android, iOS, Windows Phone and Mac platforms. Because Xamarin is written in C\#, it can integrate easily with the Windows SDK for the development of Windows Desktop Applications. This allows business layer logic to be shared across the different platform implementations, essentially having one main codebase rather than one per platform.

\begin{figure}[htbp]
	\centering
		\includegraphics[width=\textwidth,height=\textheight,keepaspectratio]{Figures/AppOverview.png}
		\rule{35em}{0.5pt}
\end{figure}

\subsection{Conclusion} 

The server application will be written in C\# and compiled using Mono, allowing it to run on either Windows or Linux server machines. The Android App will also be written in C\# using the Xamarin framework, allowing the codebase to be easily integrated with future platforms.

%----------------------------------------------------------------------------------------

\section{Communication}

The system requires two types of communication based on the requirements, urgent and direct. It is very important to differentiate between these two when targeting mobile devices, due to the following issues:
\begin{itemize}
	\item Bandwidth is limited
	\item Devices aren't always online
	\item Connections are unreliable
	\item Constant communication can cause excessive power consumption
\end{itemize}

Due to these issues, a constant connection to a remote server is not possible for long periods of time. This means that the communication must be separated into two separate components, each being useful for different scenarios.

\subsection{Push Communication}

Notification messages, also known as push notifications, are a way of sending a short notification message to a device. This is useful for sending urgent messages when the application is not being used, prompting the user for input.

There are however, limitations of this type of communication:

\begin{itemize}
	\item Only small messages are allowed to be sent
	\item The server does not know when the message has been received
	\item Server to client messages only
\end{itemize}

Push communication can be seen as a 'answering machine service'. The server leaves a short message for the device to be received at some point in the future. If the device is offline, it will receive the message as soon as it comes online again, making it a good system for unreliable connections.

Push notifications should be kept short and only contain enough data to notify the application that it needs to connect to the server for more information.

Due to these limitations, the system will only use push notifications for the following scenarios:

\begin{itemize}
	\item A sooner appointment is available for the patient
	\item An appointment has been cancelled and/or needs rescheduling
	\item Information about an appointment has been changed
\end{itemize}

\subsubsection{Google Cloud Messaging (Android)}

Google Cloud Messaging is the Android service for sending push notifications. It has a message limit of 4kb and requires the Google Play Store to function. It is a free service, however it has a daily limitation on how many notifications can be sent from a single application.

\begin{figure}[htbp]
	\centering
\includegraphics[width=10cm,height=10cm,keepaspectratio]{Figures/PushNote.png}
		\rule{35em}{0.5pt}
	\caption[Google Cloud Messaging]{Google Cloud Messaging}
	\label{fig:PushNote}
\end{figure}

By using this service, messages are queued and sent to the device as soon as it is available, prompting the user of some kind of notification relating to the application.

\subsubsection{Apple Push Notifications Gateway (iOS)}

Apple also have a push notification service. It has a message limit of just 256 bytes.

\subsection{Direct Communication}

After a notification has been received, or simply through using the applications features, the device will require direct communication with the server. It will require communication to:

\begin{itemize}
	\item Create or Reschedule an appointment.
	\item Request information about an appointment
	\item Request reminders about an appointment
	\item Requesting the database encryption key
\end{itemize}

\subsubsection{RESTful Web Services}

To be written.

\begin{figure}[htbp]
	\centering
\includegraphics[width=10cm,height=10cm,keepaspectratio]{Figures/rest.png}
		\rule{35em}{0.5pt}
	\caption[RESTful Web Services cross-platform communication]{RESTful Web Services cross-platform communication}
	\label{fig:rest}
\end{figure}


%----------------------------------------------------------------------------------------

\section{Data}

Appointment data will be cached on the mobile so that communication is minimised. Data will be stored in an sqlite database because it is built into both the Android and iOS operating systems, making it available on all target devices.

\subsection{Security}

Data security is an issue because we are dealing with sensitive patient data. Because of this, the database must be encrypted.

%----------------------------------------------------------------------------------------
 
% Chapter Template

\chapter{Implementation} % Main chapter title

\label{Chapter5} % Change X to a consecutive number; for referencing this chapter elsewhere, use \ref{ChapterX}

\lhead{Chapter 5. \emph{Implementation}} % Change X to a consecutive number; this is for the header on each page - perhaps a shortened title

%----------------------------------------------------------------------------------------

\section{Implementation Process}

To be written

\section{System Architecture}
\subsection{Design Patterns}
\subsection{Communication Methods}
\subsection{User Interface and Storyboards}

\section{Server Implementation}
\subsection{Data Access Layer}
\subsection{Data Layer}
\subsection{Business Layer}
\subsection{Application Layer}

\section{Client Implementation}
\subsection{Data Access Layer}
\subsection{Data Layer}
\subsection{Business Layer}
\subsection{Application Layer}

%---------------------------------------------------------------------------------------- 
% Chapter Template

\chapter{Testing and Evaluation} % Main chapter title

\label{Chapter6} % Change X to a consecutive number; for referencing this chapter elsewhere, use \ref{ChapterX}

\lhead{Chapter 6. \emph{Testing and Evaluation}} % Change X to a consecutive number; this is for the header on each page - perhaps a shortened title

%----------------------------------------------------------------------------------------
To be written

\section{Testing}
\subsection{Methodology}
\subsection{Result}
\subsection{Conclusion}
\section{Evaluation}
\subsection{User Evaluation}
\subsection{Methodology}
\subsubsection{Controlled Observation}
\subsubsection{Interviews}
\subsection{conclusion}
%---------------------------------------------------------------------------------------- 
% Chapter Template

\chapter{Project Conclusion} % Main chapter title

\label{Chapter7} % Change X to a consecutive number; for referencing this chapter elsewhere, use \ref{ChapterX}

\lhead{Chapter 7. \emph{Project Conclusion}} % Change X to a consecutive number; this is for the header on each page - perhaps a shortened title

%----------------------------------------------------------------------------------------
\section{Summary}

This project aimed to design a prototype system that helped automate the booking and management of medical appointments. The project achieved this by providing direct communication links

\section{Limitations}

\section{Future Extensions}

By increasing communication with the patients, many useful extensions could be investigated in the future that were not feasible given the time restrictions of this project.

\section{Final Conclusion}

%---------------------------------------------------------------------------------------- 

%----------------------------------------------------------------------------------------
%	THESIS CONTENT - APPENDICES
%----------------------------------------------------------------------------------------

\addtocontents{toc}{\vspace{2em}} % Add a gap in the Contents, for aesthetics

\appendix % Cue to tell LaTeX that the following 'chapters' are Appendices

% Include the appendices of the thesis as separate files from the Appendices folder
% Uncomment the lines as you write the Appendices

% Appendix Template

\chapter{Personal Reflection} % Main appendix title

\label{AppendixA} % Change X to a consecutive letter; for referencing this appendix elsewhere, use \ref{AppendixX}

\lhead{Appendix A. \emph{Personal Reflection}} % Change X to a consecutive letter; this is for the header on each page - perhaps a shortened title

%talk about why you chose the project, 
When first faced with my final year project, I was overwhelmed with choice. Throughout my university career, I have rarely had the freedom to choose precisely what I wanted to learn. It is therefore very tempting to pick an area that you already know well, with very little learning benefit.

To overcome this, I set out a list of personal goals that I wanted to achieve by doing a project, which helped me narrow my choices:

\begin{itemize}
	\item I wanted to learn how to build mobile applications.
	\item I wanted to learn how to build an adaptable, scalable server.
	\item I wanted to provide a tool that would potentially help people.
\end{itemize}

It became clear after reading through project choices submitted by lecturers that none of them would help me obtain my goals. Thankfully, with the help of my supervisor, I was able to draft out a suitable project that was suitable enough to achieve both the universities and my own goals.

%talk about project planning, lack of deadlines
The second challenge I faced was project planning. I was not used to the lack of deadlines, and it took a long time for me to get started on my project. I overcame this problem through the use of personal deadlines. Although creating a Gantt chart helped keep track of the project's progress, I did not keep to it initially. I discovered the hard way by falling behind, and I spent many painful hours catching up to my initial schedule.

Another issue I faced with the project planning was encountering entire tasks. Although I had split my project into sub tasks such as implementation, it seemed very daunting when first facing them. To overcome this, I used project management tools like 'Trello' to split tasks down into subtasks, leave comments that I could remember and plan features out. This helped overcome the initial stress involved when planning specific parts of the project, making todo lists and remembering issues that occured for when you write the report. I found this far more valuable than keeping a journal as the user interface is very flexible and appealing.

The aspect of the project I enjoyed the most was the implementation. Although I had outlined things that I wanted to learn by doing this project, I learned far more during implementation due to issues that I had not even considered. An example of this was my problems with time synchronisation across multiple regions around the world, requiring me to look into solutions to this and learning about the various practices involved in geographical programming.

Besides implementation, this project has taught me invaluable skills that I will no doubt be utilising for years to come. I learnt how to formulate a report, conduct evaluation on research literature, software testing, communicating with various entities to get project resources, user based evaluation and many more skills involved.

Ultimately, my final year project was the highlight of my degree for me. It has been the greatest learning opportunity and challenge that I have faced in my university career, and I am grateful for having the opportunity to participate in it.

Finally, I will summarise my findings to compose this list of recommendations for future students:

\begin{itemize}
	\item Don't pick a project that sounds easy, choose what \bf{you} want to learn first, then find a project that fits your learning goals.
	\item Make personal deadlines to achieve project tasks on time. Plan accordingly in advance!
	\item Use tools such as Trello to plan your implementation, write down any issues you encounter so that you can discuss them in your report.
	\item Plan your evaluation in advance, make sure you've finished your implementation with plenty time to spare!
	\item Backup your project in multiple places, Github and dropbox are great tools!
	\item Enjoy your project and make it what you want it to be.
	\item Give yourself extra time to deal with unforeseen errors, they are a certainty.
\end{itemize}

% Appendix Template

\chapter{External Resources and Materials} % Main appendix title

\label{AppendixB} % Change X to a consecutive letter; for referencing this appendix elsewhere, use \ref{AppendixX}

\lhead{Appendix B. \emph{External Resources and Materials}} % Change X to a consecutive letter; this is for the header on each page - perhaps a shortened title

\begin{itemize}
	\item Use of the Xamarin Framework to develop the Android application obtainable at:\\ \url{http://xamarin.com/}
	\item Use of the ASP.Net Framework to develop the Server application obtainable at:\\ \url{http://asp.net}
	\item Use of the Entity Framework library to crate and manage SQL databases in C\#, available at: \url{http://msdn.microsoft.com/en-gb/data/ef.aspx}
	\item Use of the Rest Sharp library for the creation of RESTful web requests and the de-serialisation of response data, available at: \url{https://github.com/restsharp/RestSharp}
	\item Use of the GCM Client library for integrating GCM with Xamarin, obtainable at:\\ \url{https://github.com/Redth/GCM.Client}
	\item Windows Azure was used to host the server application and the database, available at:
	\url{http://medibook.azurewebsites.net/}
	\item Use of the Masters/Doctoral Thesis Latex Template for the creation of this report, obtainable at: \url{http://www.latextemplates.com/template/masters-doctoral-thesis}
\end{itemize}
% Appendix Template

\chapter{How ethical issues are addressed} % Main appendix title

\label{AppendixC} % Change X to a consecutive letter; for referencing this appendix elsewhere, use \ref{AppendixX}

\lhead{Appendix C. \emph{How ethical issues are addressed}} % Change X to a consecutive letter; this is for the header on each page - perhaps a shortened title

\section{Introduction}

This section describes the ethical issues involved throughout the project. I also propose techniques for solving and minimising these issues.

\section{Project background}

A few ethical issues arise from the projects aims and objectives. Firstly, personal data will be used and stored in order to optimise the projects aims. Secondly, the project aims to optimise the scheduling process, which could make some employees jobs redundant.

\subsection{Personal Data}

Personal data will be stored and used both locally (on the mobile device) and on an external server to try and optimise scheduling software. It may also be required to transmit this data regularly to keep the system running effectively.

This brings into security issues, as the data could be very valuable to certain individuals for marketing or other purposes. In order to solve this, several areas should considered:

\begin{itemize}
	\item Patients must be informed about what data is being stored and why
	\item The system should be as secure as possible with the current technology
	\item The data should be as anonymous as possible, with no individual besides its owner having access to it.
\end{itemize}

\subsection{Employee Downsizing}

The system aims to optimise the scheduling process and would replace a lot of the manual labour involved. It is therefore possible that the system would make some employees redundant.

Although this system aims to optimise the current system, it would not be able to replace it entirely, and so not all jobs would be lost. Also, this is only a short term problem, and with more training, employees could be allocated elsewhere.

\section{Testing and Evaluation}

In order to evaluate this project, I plan to perform user based assessments.

\subsection{User Based Assessments}

When performing user based assessments, care must be taken to ensure that participants are well informed of their rights, what the project is about and how their feedback would be stored and used. The participants data would also need to be stored anonymously to preserve confidentiality.

To ensure this is done correctly, the following steps will be taken:

\begin{enumerate}
	\item The nature of the project will be explained to the user.
	\item The user will be informed that their opinions will be used and stored both anonymously and securely.
	\item The user will be informed that they can stop the assessment at any time
	\item The user will be asked to sign a consent form to confirm that they have agreed to take part and that their data can be used to evaluate the project
	\item The data will be collected and stored anonymously
\end{enumerate}
% Appendix Template

\chapter{Regression Test} % Main appendix title

\label{AppendixD} % Change X to a consecutive letter; for referencing this appendix elsewhere, use \ref{AppendixX}

\lhead{Appendix D. \emph{Regression Test}} % Change X to a consecutive letter; this is for the header on each page - perhaps a shortened title

\begin{table}[h]
\centering
\resizebox{\textwidth}{!}{%
\begin{tabular}{|m{1cm}|m{3cm}|m{4cm}|m{6cm}|m{2cm}|m{6cm}|}
\hline
Test \# & Feature Category              & Feature Type                 & Action                                                                                                                    & Pass / Fail & Comments / Fault Description                                                                                                                               \\ \hline
1       & Login                         & Login Button                 & Logs the user in and progresses to the appointment list screen                                                            & Pass        &                                                                                                                                                            \\ \hline
2       & Login                         & Register Button              & Registers the user using the entered username and password, logs the user in and progresses to the appointment screen     & Pass        &                                                                                                                                                            \\ \hline
3       & Login                         & Username/Password Text-Field & Tap on either field brings up the android keyboard. Input is correctly passed to the field and used as login credentials. & Pass        & Keyboard covers the buttons partially, however pressing back hides the keyboard                                                                            \\ \hline
4       & Appointment List              & List                         & Downloads correct appointments for the user and displays type, status and scheduled time.                                 & Pass        &                                                                                                                                                            \\ \hline
5       & Appointment List              & List Item                    & Tapping on a list item should open the appointment information screen.                                                    & Pass        & Colour should match buttons so patients know to click on items.                                                                                            \\ \hline
6       & Appointment List              & Refresh Button               & Tapping re-downloads appointments. Shows animation while downloading and correctly resets when done.                      & Fail        & Shows on the notifications tab.                                                                                                                            \\ \hline
7       & Appointment List              & Logout Button                & Logs the user out returns to the login screen                                                                             & Pass        &                                                                                                                                                            \\ \hline
8       & Notifications                 & Notification                 & Device Receives notification from server                                                                                  & Fail        & Sometimes notification is not received.Cause: The device Id is sent to the server after some notifications are sent resulting in them not to be delivered. \\ \hline
9       & Notifications List            & List                         & Notifications                                                                                                             & Pass        &                                                                                                                                                            \\ \hline
10      & Notification / Appointment List & Tab                          & Tapping on the tabs switch between notification and appointment lists.                                                    & Pass        &                                                                                                                                                            \\ \hline
11      & Information screen            & Doctors Information          & Loads a portrait image, name and type of doctor.                                                                          & Pass        & Doctors image sometimes slow to load.                                                                                                                      \\ \hline
12      & Information Screen            & Doctor/ Clinic Number        & Shows correct number for the doctor or clinic and tapping on it opens phone dialer                                        & Fail        & Tapping doesn't open dialer.Cause: Missing button Id.                                                                                                    \\ \hline
13      & Information Screen            & Map Button                   & Shows the map screen                                                                                                      & Pass        &                                                                                                                                                            \\ \hline
14      & Calendar                      & Add to Calendar Button       & Only visible if appointment is not added to calendar. Disabled if appointment is not scheduled. Tapping adds to calendar  & Pass        &                                                                                                                                                            \\ \hline
15      & Calendar                      & Remove from Calendar Button  & Only visible if appointment is added to calendar. Tapping removes from calendar                                           & Fail        & Does not remove from the phone's calendar correctlyCause: Incorrect SQL selection query                                                                  \\ \hline
16      & Information Screen            & Schedule Appointment Button  & Shows button text as re-schedule if appointment already scheduled. Tapping button opens scheduling screen.                & Pass        &                                                                                                                                                            \\ \hline
17      & Information Screen            & Cancel Appointment Button    & Cancels the appointment                                                                                                   & Fail        & Cancels the appointment correctly however does not update the appointment list which still shows it as scheduled.                                          \\ \hline
18      & Information Screen            & Map Button                   & Opens the map screen                                                                                                      & Pass        &                                                                                                                                                            \\ \hline
19      & Information Screen            & Scheduled Time               & Shows the appointment time                                                                                                & Fail        & Shows the incorrect time (1 hour before).Cause: Time is sent in UTC format and is wrong time zone.                                                         \\ \hline
20      & Map                           & Map                          & Shows appointment Location                                                                                                & Pass        & Only works in release.                                                                                                                                     \\ \hline
21      & Schedule Appointment          & Schedule Appointment         & Schedules appointment correctly, giving three options if time is taken and allowing user to specify a time.               & Pass        &                                                                                                                                                            \\ \hline
\end{tabular}
}
\end{table}
% Appendix Template

\chapter{User Evaluation} % Main appendix title

\label{AppendixE} % Change X to a consecutive letter; for referencing this appendix elsewhere, use \ref{AppendixX}

\lhead{Appendix E. \emph{User Evaluation}} % Change X to a consecutive letter; this is for the header on each page - perhaps a shortened title

\section{Introduction}

The purpose of this evaluation is to determine the usability and success of the mobile application 'MediBook'. You are asked to carry out all of the tasks listed below and then answer a short survey to describe your experience. Your responses and feedback gained from this survey will aid in the evaluation of the project. You will also be observed during the evaluation to collect further data.

All data collected from this activity is confidential and your identity will be kept anonymous. Thank you for agreeing to participate in this evaluation.

You will be given a test account with appointments already setup for you. Once you have received this information, please carry out the following tasks.

\section{Task List}

\begin{enumerate}
	\item Find the MediBook icon in the app library and open the app.
	\item Login with the test account and find the appointment list.
	\item Try to find the name of the location where the first appointment occurs.
	\item If you have found this location, try and open it on the map screen.
	\item Return to the appointment list.
	\item Select the second appointment and try to schedule it for 15:00pm today.
	\item If you successfully scheduled the appointment, try and find the message displayed in the notification you just received.
	\item Return to the app and select the third appointment. Try and add it to the phone's calendar.
	\item Check that the appointment is successfully inserted into the phones calendar.
	\item Return to the app appointment list and select the logout button.
	\item Once returned to the login screen, close the app.
\end{enumerate}
% Appendix Template

\chapter{User Evaluation Questionnaire} % Main appendix title

\label{AppendixF} % Change X to a consecutive letter; for referencing this appendix elsewhere, use \ref{AppendixX}

\lhead{Appendix F. \emph{User Evaluation Questionnaire}} % Change X to a consecutive letter; this is for the header on each page - perhaps a shortened title

1) Are you satisfied with the current methods of managing medical appointments?

\begin{itemize}
	\item 6 - Extremely satisfied
	\item 5 - Moderately satisfied
	\item 4 - Slightly satisfied
	\item 3 - Slightly dissatisfied
	\item 2 - Moderately dissatisfied
	\item 1 - Extremely dissatisfied
\end{itemize}

2) How do you currently manage your medical appointments?
\\[1in]

3) Do you have access to a smart device of your own with internet access? (Please circle those that apply)

\begin{itemize}
	\item I have access to an Android device.
	\item I have access to an iOS device.
	\item I have access to a Windows Phone device.
	\item I do not have access to a smart device.
\end{itemize}

\newpage

4) How easy to use was the application? (Please circle one)

\begin{itemize}
	\item 6 - Extremely easy
	\item 5 - Moderately easy
	\item 4 - Slightly easy
	\item 3 - Slightly difficult
	\item 2 - Moderately difficult
	\item 1 - Extremely difficult
\end{itemize}

5) How useful would this application be to you in the future? (Please circle one)

\begin{itemize}
	\item 6 - Extremely useful
	\item 5 - Moderately useful
	\item 4 - Slightly Useful
	\item 3 - Not very useful
	\item 2 - Rarely useful
	\item 1 - I would never use it
\end{itemize}

6) How pleasing is the design of the application (Please circle one)

\begin{itemize}
	\item 6 - Extremely pleasing
	\item 5 - Moderately pleasing
	\item 4 - Slightly pleasing
	\item 3 - Slightly displeasing
	\item 2 - Moderately displeasing
	\item 1 - Extremely displeasing
\end{itemize}

7) Did you have any problems carrying out any of the tasks?
\\[1in]

8) Do you think the application's reminder notifications would help you fulfil appointment requirements?
\\[1in]

9) Which features did you find useful?
\\[1in]

10) Was the application missing any features you would find useful?
\\[1in]

11) Do you have any suggestions as to how the application could be improved.
\\[1in]
% Appendix Template

\chapter{Project Resources} % Main appendix title

\label{AppendixG} % Change X to a consecutive letter; for referencing this appendix elsewhere, use \ref{AppendixX}

\lhead{Appendix G. \emph{Project Resources}} % Change X to a consecutive letter; this is for the header on each page - perhaps a shortened title

\section{Code and Mobile Application Download}

The project code can be found at the following address:

\url{https://github.com/andrewmunro/Final-Year-Project/tree/master/Prototype}

The mobile application apk can be downloaded at the following address:

\url{https://www.dropbox.com/s/4lzrs7gnjauw7sj/MediBook.Client.Android.Release.apk}

The hosted server application can be found at the following address:

\url{http://medibook.azurewebsites.net/}

\section{Mobile Application Screens}

\begin{figure}[htbp]
	\centering
\includegraphics[width=\textwidth,height=\textheight,keepaspectratio]{Figures/screens/HomeLoginScreen.png}
		\rule{35em}{0.5pt}
	\caption[Login Screen (Left), Home Screen (Right)]{Login Screen (Left), Home Screen (Right)}
	\label{fig:loginhomescreen}
\end{figure}

\begin{figure}[htbp]
	\centering
\includegraphics[width=\textwidth,height=\textheight,keepaspectratio]{Figures/screens/AppointmentScheduleScreen.png}
		\rule{35em}{0.5pt}
	\caption[Appointment Information Screen (Left), Appointment Time Picker Screen (Right)]{Appointment Information Screen (Left), Appointment Time Picker Screen (Right)}
	\label{fig:appointmentschedulescreen}
\end{figure}

\begin{figure}[htbp]
	\centering
\includegraphics[width=\textwidth,height=\textheight,keepaspectratio]{Figures/screens/ConfirmScheduleScreen.png}
		\rule{35em}{0.5pt}
	\caption[Conflicting Appointment Screen (Left), Confirm Appointment Time Screen (Right)]{Conflicting Appointment Screen (Left), Confirm Appointment Time Screen (Right)}
	\label{fig:conflictappointscreen}
\end{figure}




\addtocontents{toc}{\vspace{2em}} % Add a gap in the Contents, for aesthetics

\backmatter

%----------------------------------------------------------------------------------------
%	BIBLIOGRAPHY
%----------------------------------------------------------------------------------------

\label{Bibliography}

\lhead{\emph{Bibliography}} % Change the page header to say "Bibliography"

\bibliographystyle{unsrtnat} % Use the "unsrtnat" BibTeX style for formatting the Bibliography

\bibliography{Bibliography} % The references (bibliography) information are stored in the file named "Bibliography.bib"

\end{document}  