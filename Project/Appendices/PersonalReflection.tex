% Appendix Template

\chapter{Personal Reflection} % Main appendix title

\label{AppendixA} % Change X to a consecutive letter; for referencing this appendix elsewhere, use \ref{AppendixX}

\lhead{Appendix A. \emph{Personal Reflection}} % Change X to a consecutive letter; this is for the header on each page - perhaps a shortened title

%talk about why you chose the project, 
When first faced with my final year project, I was overwhelmed with choice. Throughout my university career, I have rarely had the freedom to choose precisely what I wanted to learn. It is therefore very tempting to pick an area that you already know well, with very little learning benefit.

To overcome this, I set out a list of personal goals that I wanted to achieve by doing a project, which helped me narrow my choices:

\begin{itemize}
	\item I wanted to learn how to build mobile applications.
	\item I wanted to learn how to build an adaptable, scalable server.
	\item I wanted to provide a tool that would potentially help people.
\end{itemize}

It became clear after reading through project choices submitted by lecturers that none of them would help me obtain my goals. Thankfully, with the help of my supervisor, I was able to draft out a suitable project that was suitable enough to achieve both the universities and my own goals.

%talk about project planning, lack of deadlines
The second challenge I faced was project planning. I was not used to the lack of deadlines, and it took a long time for me to get started on my project. I overcame this problem through the use of personal deadlines. Although creating a Gantt chart helped keep track of the project's progress, I did not keep to it initially. I discovered the hard way by falling behind, and I spent many painful hours catching up to my initial schedule.

Another issue I faced with the project planning was encountering entire tasks. Although I had split my project into sub tasks such as implementation, it seemed very daunting when first facing them. To overcome this, I used project management tools like 'Trello' to split tasks down into subtasks, leave comments that I could remember and plan features out. This helped overcome the initial stress involved when planning specific parts of the project, making todo lists and remembering issues that occured for when you write the report. I found this far more valuable than keeping a journal as the user interface is very flexible and appealing.

The aspect of the project I enjoyed the most was the implementation. Although I had outlined things that I wanted to learn by doing this project, I learned far more during implementation due to issues that I had not even considered. An example of this was my problems with time synchronisation across multiple regions around the world, requiring me to look into solutions to this and learning about the various practices involved in geographical programming.

Besides implementation, this project has taught me invaluable skills that I will no doubt be utilising for years to come. I learnt how to formulate a report, conduct evaluation on research literature, software testing, communicating with various entities to get project resources, user based evaluation and many more skills involved.

Ultimately, my final year project was the highlight of my degree for me. It has been the greatest learning opportunity and challenge that I have faced in my university career, and I am grateful for having the opportunity to participate in it.

Finally, I will summarise my findings to compose this list of recommendations for future students:

\begin{itemize}
	\item Don't pick a project that sounds easy, choose what \bf{you} want to learn first, then find a project that fits your learning goals.
	\item Make personal deadlines to achieve project tasks on time. Plan accordingly in advance!
	\item Use tools such as Trello to plan your implementation, write down any issues you encounter so that you can discuss them in your report.
	\item Plan your evaluation in advance, make sure you've finished your implementation with plenty time to spare!
	\item Backup your project in multiple places, Github and dropbox are great tools!
	\item Enjoy your project and make it what you want it to be.
	\item Give yourself extra time to deal with unforeseen errors, they are a certainty.
\end{itemize}
